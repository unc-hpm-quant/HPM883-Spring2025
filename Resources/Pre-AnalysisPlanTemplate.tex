% Options for packages loaded elsewhere
\PassOptionsToPackage{unicode}{hyperref}
\PassOptionsToPackage{hyphens}{url}
\PassOptionsToPackage{dvipsnames,svgnames,x11names}{xcolor}
%
\documentclass[
  letterpaper,
  DIV=11,
  numbers=noendperiod]{scrartcl}

\usepackage{amsmath,amssymb}
\usepackage{iftex}
\ifPDFTeX
  \usepackage[T1]{fontenc}
  \usepackage[utf8]{inputenc}
  \usepackage{textcomp} % provide euro and other symbols
\else % if luatex or xetex
  \usepackage{unicode-math}
  \defaultfontfeatures{Scale=MatchLowercase}
  \defaultfontfeatures[\rmfamily]{Ligatures=TeX,Scale=1}
\fi
\usepackage{lmodern}
\ifPDFTeX\else  
    % xetex/luatex font selection
\fi
% Use upquote if available, for straight quotes in verbatim environments
\IfFileExists{upquote.sty}{\usepackage{upquote}}{}
\IfFileExists{microtype.sty}{% use microtype if available
  \usepackage[]{microtype}
  \UseMicrotypeSet[protrusion]{basicmath} % disable protrusion for tt fonts
}{}
\makeatletter
\@ifundefined{KOMAClassName}{% if non-KOMA class
  \IfFileExists{parskip.sty}{%
    \usepackage{parskip}
  }{% else
    \setlength{\parindent}{0pt}
    \setlength{\parskip}{6pt plus 2pt minus 1pt}}
}{% if KOMA class
  \KOMAoptions{parskip=half}}
\makeatother
\usepackage{xcolor}
\setlength{\emergencystretch}{3em} % prevent overfull lines
\setcounter{secnumdepth}{-\maxdimen} % remove section numbering
% Make \paragraph and \subparagraph free-standing
\makeatletter
\ifx\paragraph\undefined\else
  \let\oldparagraph\paragraph
  \renewcommand{\paragraph}{
    \@ifstar
      \xxxParagraphStar
      \xxxParagraphNoStar
  }
  \newcommand{\xxxParagraphStar}[1]{\oldparagraph*{#1}\mbox{}}
  \newcommand{\xxxParagraphNoStar}[1]{\oldparagraph{#1}\mbox{}}
\fi
\ifx\subparagraph\undefined\else
  \let\oldsubparagraph\subparagraph
  \renewcommand{\subparagraph}{
    \@ifstar
      \xxxSubParagraphStar
      \xxxSubParagraphNoStar
  }
  \newcommand{\xxxSubParagraphStar}[1]{\oldsubparagraph*{#1}\mbox{}}
  \newcommand{\xxxSubParagraphNoStar}[1]{\oldsubparagraph{#1}\mbox{}}
\fi
\makeatother


\providecommand{\tightlist}{%
  \setlength{\itemsep}{0pt}\setlength{\parskip}{0pt}}\usepackage{longtable,booktabs,array}
\usepackage{calc} % for calculating minipage widths
% Correct order of tables after \paragraph or \subparagraph
\usepackage{etoolbox}
\makeatletter
\patchcmd\longtable{\par}{\if@noskipsec\mbox{}\fi\par}{}{}
\makeatother
% Allow footnotes in longtable head/foot
\IfFileExists{footnotehyper.sty}{\usepackage{footnotehyper}}{\usepackage{footnote}}
\makesavenoteenv{longtable}
\usepackage{graphicx}
\makeatletter
\newsavebox\pandoc@box
\newcommand*\pandocbounded[1]{% scales image to fit in text height/width
  \sbox\pandoc@box{#1}%
  \Gscale@div\@tempa{\textheight}{\dimexpr\ht\pandoc@box+\dp\pandoc@box\relax}%
  \Gscale@div\@tempb{\linewidth}{\wd\pandoc@box}%
  \ifdim\@tempb\p@<\@tempa\p@\let\@tempa\@tempb\fi% select the smaller of both
  \ifdim\@tempa\p@<\p@\scalebox{\@tempa}{\usebox\pandoc@box}%
  \else\usebox{\pandoc@box}%
  \fi%
}
% Set default figure placement to htbp
\def\fps@figure{htbp}
\makeatother
% definitions for citeproc citations
\NewDocumentCommand\citeproctext{}{}
\NewDocumentCommand\citeproc{mm}{%
  \begingroup\def\citeproctext{#2}\cite{#1}\endgroup}
\makeatletter
 % allow citations to break across lines
 \let\@cite@ofmt\@firstofone
 % avoid brackets around text for \cite:
 \def\@biblabel#1{}
 \def\@cite#1#2{{#1\if@tempswa , #2\fi}}
\makeatother
\newlength{\cslhangindent}
\setlength{\cslhangindent}{1.5em}
\newlength{\csllabelwidth}
\setlength{\csllabelwidth}{3em}
\newenvironment{CSLReferences}[2] % #1 hanging-indent, #2 entry-spacing
 {\begin{list}{}{%
  \setlength{\itemindent}{0pt}
  \setlength{\leftmargin}{0pt}
  \setlength{\parsep}{0pt}
  % turn on hanging indent if param 1 is 1
  \ifodd #1
   \setlength{\leftmargin}{\cslhangindent}
   \setlength{\itemindent}{-1\cslhangindent}
  \fi
  % set entry spacing
  \setlength{\itemsep}{#2\baselineskip}}}
 {\end{list}}
\usepackage{calc}
\newcommand{\CSLBlock}[1]{\hfill\break\parbox[t]{\linewidth}{\strut\ignorespaces#1\strut}}
\newcommand{\CSLLeftMargin}[1]{\parbox[t]{\csllabelwidth}{\strut#1\strut}}
\newcommand{\CSLRightInline}[1]{\parbox[t]{\linewidth - \csllabelwidth}{\strut#1\strut}}
\newcommand{\CSLIndent}[1]{\hspace{\cslhangindent}#1}

\KOMAoption{captions}{tableheading}
\makeatletter
\@ifpackageloaded{tcolorbox}{}{\usepackage[skins,breakable]{tcolorbox}}
\@ifpackageloaded{fontawesome5}{}{\usepackage{fontawesome5}}
\definecolor{quarto-callout-color}{HTML}{909090}
\definecolor{quarto-callout-note-color}{HTML}{0758E5}
\definecolor{quarto-callout-important-color}{HTML}{CC1914}
\definecolor{quarto-callout-warning-color}{HTML}{EB9113}
\definecolor{quarto-callout-tip-color}{HTML}{00A047}
\definecolor{quarto-callout-caution-color}{HTML}{FC5300}
\definecolor{quarto-callout-color-frame}{HTML}{acacac}
\definecolor{quarto-callout-note-color-frame}{HTML}{4582ec}
\definecolor{quarto-callout-important-color-frame}{HTML}{d9534f}
\definecolor{quarto-callout-warning-color-frame}{HTML}{f0ad4e}
\definecolor{quarto-callout-tip-color-frame}{HTML}{02b875}
\definecolor{quarto-callout-caution-color-frame}{HTML}{fd7e14}
\makeatother
\makeatletter
\@ifpackageloaded{caption}{}{\usepackage{caption}}
\AtBeginDocument{%
\ifdefined\contentsname
  \renewcommand*\contentsname{Table of contents}
\else
  \newcommand\contentsname{Table of contents}
\fi
\ifdefined\listfigurename
  \renewcommand*\listfigurename{List of Figures}
\else
  \newcommand\listfigurename{List of Figures}
\fi
\ifdefined\listtablename
  \renewcommand*\listtablename{List of Tables}
\else
  \newcommand\listtablename{List of Tables}
\fi
\ifdefined\figurename
  \renewcommand*\figurename{Figure}
\else
  \newcommand\figurename{Figure}
\fi
\ifdefined\tablename
  \renewcommand*\tablename{Table}
\else
  \newcommand\tablename{Table}
\fi
}
\@ifpackageloaded{float}{}{\usepackage{float}}
\floatstyle{ruled}
\@ifundefined{c@chapter}{\newfloat{codelisting}{h}{lop}}{\newfloat{codelisting}{h}{lop}[chapter]}
\floatname{codelisting}{Listing}
\newcommand*\listoflistings{\listof{codelisting}{List of Listings}}
\makeatother
\makeatletter
\makeatother
\makeatletter
\@ifpackageloaded{caption}{}{\usepackage{caption}}
\@ifpackageloaded{subcaption}{}{\usepackage{subcaption}}
\makeatother

\usepackage{bookmark}

\IfFileExists{xurl.sty}{\usepackage{xurl}}{} % add URL line breaks if available
\urlstyle{same} % disable monospaced font for URLs
\hypersetup{
  pdftitle={Pre-Analysis Plan: Title of the Study},
  pdfauthor={Author's Name},
  colorlinks=true,
  linkcolor={blue},
  filecolor={Maroon},
  citecolor={Blue},
  urlcolor={Blue},
  pdfcreator={LaTeX via pandoc}}


\title{Pre-Analysis Plan: Title of the Study}
\author{Author's Name}
\date{Invalid Date}

\begin{document}
\maketitle

\renewcommand*\contentsname{Table of contents}
{
\hypersetup{linkcolor=}
\setcounter{tocdepth}{3}
\tableofcontents
}

\begin{tcolorbox}[enhanced jigsaw, breakable, colback=white, bottomtitle=1mm, toptitle=1mm, toprule=.15mm, colframe=quarto-callout-note-color-frame, colbacktitle=quarto-callout-note-color!10!white, rightrule=.15mm, left=2mm, bottomrule=.15mm, leftrule=.75mm, title=\textcolor{quarto-callout-note-color}{\faInfo}\hspace{0.5em}{Note}, arc=.35mm, titlerule=0mm, opacityback=0, opacitybacktitle=0.6, coltitle=black]

Template for a Pre-Analysis Plan (PAP) for a randomized experiment. This
is modified from an original template created by Alejandro Ganimian,
available
\href{https://raw.githubusercontent.com/BITSS/INSP2017/master/2-Reg-and-PAP/pre-analysis\%20plan\%20template.tex}{here}.

Other Helpful Resources:

For guidance on pre-analysis plans, refer to

\begin{itemize}
\tightlist
\item
  the World Bank's DIME Wiki:
  \href{https://dimewiki.worldbank.org/Pre-Analysis_Plan}{Pre-Analysis
  Plan - DIME Wiki}
\item
  The J-Pal Research Resources Website:
  \href{https://www.povertyactionlab.org/resource/pre-analysis-plans}{J-Pal
  Research Resources}
\end{itemize}

For examples of pre-analysis plans, explore the AEA's RCT Registry:
\href{https://www.socialscienceregistry.org/}{AEA RCT Registry}. Here
are some of mine:

\begin{itemize}
\tightlist
\item
  \href{https://www.socialscienceregistry.org/trials/411}{Pay by Design
  Trial}
\item
  \href{https://www.socialscienceregistry.org/trials/1469}{Anemia P4P
  Trial}
\end{itemize}

\end{tcolorbox}

\section{Introduction}\label{introduction}

\subsection{Abstract}\label{abstract}

\begin{itemize}
\tightlist
\item
  In 1-2 sentences, what does the study entail?
\item
  In 1-2 sentences, why is this study important/relevant?
\end{itemize}

\subsection{Motivation}\label{motivation}

\begin{itemize}
\tightlist
\item
  What is the main problem/question motivating the study?
\item
  How has this problem/question been addressed thus far?
\item
  How is this study different from prior research on this
  problem/question?
\item
  Why is the context that you have chosen for this study appropriate?
\end{itemize}

\subsection{Research Questions}\label{research-questions}

\begin{itemize}
\tightlist
\item
  What are the main research questions the study seeks to answer?
\end{itemize}

\section{Research Strategy}\label{research-strategy}

\subsection{Sampling}\label{sampling}

\subsubsection{Sampling Frame}\label{sampling-frame}

\begin{itemize}
\tightlist
\item
  What is the eligible population for the study?

  \begin{itemize}
  \tightlist
  \item
    What are the main characteristics of this population?
  \end{itemize}
\item
  What is the expected sample for the study?

  \begin{itemize}
  \tightlist
  \item
    What is the expected sample size?
  \item
    How does the expected sample differ from the population?
  \end{itemize}
\end{itemize}

\subsubsection{Statistical Power}\label{statistical-power}

\begin{itemize}
\tightlist
\item
  What is the effect size you will be able to detect?

  \begin{itemize}
  \tightlist
  \item
    What are your assumptions about your alpha-level?
  \item
    What are your assumptions about your statistical power?
  \item
    What are your assumptions about variability in your effect size?
  \item
    How many sites will you have?
  \item
    How many people will you have in each site?
  \item
    What share of the variance do you expect to predict with your
    covariates?
  \end{itemize}
\item
  How sensitive is your effect size to changes in your parameters?
\end{itemize}

\subsubsection{Assignment to Treatment}\label{assignment-to-treatment}

\begin{itemize}
\tightlist
\item
  How will individuals be assigned to treatment and control conditions?
\item
  What is the source of exogenous variation in your study?
\end{itemize}

\subsubsection{Attrition from the
Sample}\label{attrition-from-the-sample}

\begin{itemize}
\tightlist
\item
  Do you anticipate any form of attrition from the sample?

  \begin{itemize}
  \tightlist
  \item
    If so, what share of the sample do you anticipate will attrit?
  \item
    On what evidence are you basing your expectations about attrition?
  \item
    How realistic are your expectations about attrition?
  \end{itemize}
\item
  What can you do to prevent/remedy sample attrition?
\item
  How does expected attrition change your power calculations?
\end{itemize}

\subsection{Fieldwork}\label{fieldwork}

\subsubsection{Instruments}\label{instruments}

\begin{itemize}
\tightlist
\item
  What data collection instruments will you employ?

  \begin{itemize}
  \tightlist
  \item
    What (groups of) indicators will each instrument cover?
  \item
    How was each instrument developed?
  \item
    Have each instrument been used before?
  \item
    If so, by whom? If not, are you piloting it?
  \item
    What are the main advantages/disadvantages of each instrument?
  \end{itemize}
\end{itemize}

\subsubsection{Data Collection}\label{data-collection}

\begin{itemize}
\tightlist
\item
  How long will the entire data collection process take from start to
  finish?
\item
  What does the data collection entail?
\item
  What steps will be taken to keep the data collected confidential at
  this stage?
\end{itemize}

\subsubsection{Data Processing}\label{data-processing}

\begin{itemize}
\tightlist
\item
  How long will data processing take from start to finish?
\item
  What does the data processing entail?
\item
  What steps will be taken to keep the processed data confidential?
\item
  Who has ownership over the processed data?
\item
  How will the data be used/stored after the study at this stage?
\end{itemize}

\section{Empirical Analysis}\label{empirical-analysis}

\subsection{Variables}\label{variables}

\begin{itemize}
\tightlist
\item
  What are the main variables of interest in your study?

  \begin{itemize}
  \tightlist
  \item
    How is each of them defined in your dataset?
  \end{itemize}
\end{itemize}

\subsection{Balancing Checks}\label{balancing-checks}

\begin{itemize}
\tightlist
\item
  How will you check balance between treatment and control groups?

  \begin{itemize}
  \tightlist
  \item
    What is the specification that you will run?
  \item
    What variables will you include in these balancing checks?
  \end{itemize}
\item
  How will you check balance between attritors and non-attritors?

  \begin{itemize}
  \tightlist
  \item
    What is the specification that you will run?
  \item
    What variables will you include in these balancing checks?
  \end{itemize}
\end{itemize}

\subsection{Treatment Effects}\label{treatment-effects}

\subsubsection{Intent to Treat}\label{intent-to-treat}

\begin{itemize}
\tightlist
\item
  How will you estimate the (causal) effect of the offer of the
  treatment?

  \begin{itemize}
  \tightlist
  \item
    What is the specification that you will run?
  \item
    What controls will you include in your specification?
  \end{itemize}
\end{itemize}

\subsubsection{Treatment on the Treated}\label{treatment-on-the-treated}

\begin{itemize}
\tightlist
\item
  How will you estimate the (causal) effect of the receipt of the
  treatment?

  \begin{itemize}
  \tightlist
  \item
    What is the specification that you will run?
  \item
    What controls will you include in your specification?
  \end{itemize}
\end{itemize}

\subsection{Heterogeneous Effects}\label{heterogeneous-effects}

\begin{itemize}
\tightlist
\item
  Which groups do you anticipate will display heterogeneous effects?
\item
  What is the broad theory of action that leads you to anticipate these
  effects?
\end{itemize}

\subsubsection{Intent to Treat}\label{intent-to-treat-1}

\begin{itemize}
\tightlist
\item
  How will you estimate the heterogeneous effects of the offer of the
  treatment?

  \begin{itemize}
  \tightlist
  \item
    What are the specifications that you will run?
  \item
    What controls will you include in your specification?
  \end{itemize}
\end{itemize}

\subsubsection{Treatment on the
Treated}\label{treatment-on-the-treated-1}

\begin{itemize}
\tightlist
\item
  How will you estimate the heterogeneous effects of the receipt of the
  treatment?

  \begin{itemize}
  \tightlist
  \item
    What are the specifications that you will run?
  \item
    What controls will you include in your specification?
  \end{itemize}
\end{itemize}

\subsection{Standard Error
Adjustments}\label{standard-error-adjustments}

\begin{itemize}
\tightlist
\item
  How will you account for clustering in your data?
\item
  How will you address false positives from multiple hypothesis testing?

  \begin{itemize}
  \tightlist
  \item
    If you plan to adjust your standard errors, what adjustment
    procedure will you use? (e.g., Family Wise Error Rate, False
    Discovery Rates, etc.)
  \item
    If you plan to aggregate multiple variables into an index, which
    variables will you aggregate and how?
  \item
    How will you deal with outcomes with limited variation?
  \end{itemize}
\end{itemize}

\section{Research Team}\label{research-team}

\begin{itemize}
\tightlist
\item
  Who are the principal investigators of this study?

  \begin{itemize}
  \tightlist
  \item
    What will each of these investigators do?
  \end{itemize}
\item
  Will there be any research assistants in this study?

  \begin{itemize}
  \tightlist
  \item
    If so, what will these research assistants do?
  \end{itemize}
\end{itemize}

\section{Deliverables}\label{deliverables}

\begin{itemize}
\tightlist
\item
  What are the main products that will result from this study?
\item
  Who will be the lead author(s) for each of these deliverables?
\end{itemize}

\section{Calendar}\label{calendar}

\begin{itemize}
\tightlist
\item
  How long will the entire study take from start to finish?
\item
  What are the different tasks/steps to be completed each week/month?
\end{itemize}

\section{Budget}\label{budget}

\begin{itemize}
\tightlist
\item
  What will each part of this study cost?
\item
  What sources of funding do you anticipate?
\end{itemize}

\section{References}\label{references}

\phantomsection\label{refs}
\begin{CSLReferences}{0}{1}
\end{CSLReferences}




\end{document}
